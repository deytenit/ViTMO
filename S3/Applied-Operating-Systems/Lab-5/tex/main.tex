\documentclass[12pt,a4paper]{article}
% \usepackage[english]{babel}
% \usepackage[utf8x]{inputenc}

\usepackage{minted}

\usepackage{graphicx} % Required for inserting images.
\usepackage[margin=25mm]{geometry}
\parskip 4.2pt  % Sets spacing between paragraphs.
\parindent 8.4pt  % Sets leading space for paragraphs.
\usepackage[font=sf]{caption} % Changes font of captions.

\usepackage{svg}
\usepackage{amsmath}
\usepackage{amsfonts}
\usepackage{amssymb}
\usepackage{siunitx}
\usepackage{verbatim}
\usepackage{hyperref} % Required for inserting clickable links.
\usepackage{natbib} % Required for APA-style citations.
\usepackage[english, russian]{babel}

\title{Управление памятью в ОС Linux}
\author{Ерёмин Владимир}

\begin{document}
\maketitle

\section{Конфигурация}
\begin{itemize}
\item Общий объем оперативной памяти: \textbf{7810200}  KiB
\item Объем раздела подкачки: \textbf{2097152} KiB
\item Размер страницы вирт. памяти: \textbf{4096}
\item Объем свободной физ. памяти в ненагруж. системе: \textbf{6728628} KiB
\item Объем свободного пр-ва в разделе подкачки в ненагруж. системе: \textbf{2097152} KiB
\end{itemize}

\begin{minted}{bash}
$ free --kibi
               total        used        free      shared  buff/cache   available
Mem:         7810200      554736     6728628        3312      526836     7019820
Swap:        2097152           0     2097152

$ getconf PAGE_SIZE
4096
\end{minted}

\section{Эксперимент №1}

\subsection{Первый этап}

Макс. размер массива: \textbf{118000000}

\subsubsection{Сообщение}

\begin{minted}[breaklines]{bash}

$ dmesg

[ 5333.821710] oom-kill:constraint=CONSTRAINT_NONE,nodemask=(null),cpuset=/, mems_allowed=0, global_oom,task_memcg=/,task=mem.bash,pid=5201,uid=1000
[ 5333.821738] Out of memory: Killed process 5201 (mem.bash) total-vm:9240292kB, anon-rss:7333220kB, file-rss:4kB, shmem-rss:0kB, UID:1000 pgtables:18116kB oom_score_adj:0
\end{minted}

\subsubsection{График}
\begin{figure}[htbp]
  \centering
  \includesvg{first-chart.svg}
  \caption{Free Mem vs. Free Swap (single)}
\end{figure}

\subsubsection{Вывод}

График демонстрирует логичную последовательность событий: по истечении физической памяти активизировался раздел подкачки, что обеспечило продолжение выполнения процесса при использовании дополнительного объема памяти. По истечении доступного ресурса подкачки произошло аварийное завершение процесса из-за исчерпания оперативной памяти (OOM). Резкое освобождение заметного объема памяти отчетливо фиксируется в конечной части графика. Стоит отметить, что раздел подкачки не мгновенно возвращает освобожденную память, что отражено в небольшом скачке на графике физической памяти.

\subsection{Второй этап}

Макс. размер массива (база): \textbf{118000000}
Макс. размер массива (копия): \textbf{60000000}

\subsubsection{Сообщение}
\begin{minted}[breaklines]{bash}

$ dmesg

[ 8353.913293] oom-kill:constraint=CONSTRAINT_NONE,nodemask=(null),cpuset=/, mems_allowed=0, global_oom,task_memcg=/, task=mem-copy.bash,pid=6080,uid=1000
[ 8353.913421] Out of memory: Killed process 6080 (mem-copy.bash) total-vm:4703584kB, anon-rss:3676908kB, file-rss:0kB, shmem-rss:0kB, UID:1000 pgtables:9252kB oom_score_adj:0

[ 8448.533098] oom-kill:constraint=CONSTRAINT_NONE,nodemask=(null),cpuset=/, mems_allowed=0, global_oom,task_memcg=/,task=mem.bash,pid=6087,uid=1000
[ 8448.533126] Out of memory: Killed process 6087 (mem.bash) total-vm:9226564kB, anon-rss:7349100kB, file-rss:0kB, shmem-rss:0kB, UID:1000 pgtables:18096kB oom_score_adj:0
\end{minted}

\subsubsection{График}
\begin{figure}[htbp]
  \centering
  \includesvg{second-chart.svg}
  \caption{Free Mem vs. Free Swap (parallel)}
\end{figure}

\subsubsection{Вывод}

В начале данного этапа наблюдается аналогия с первым этапом, однако это происходит быстрее. Это обусловлено наличием двух параллельно выполняющихся процессов, оба из которых требуют выделения памяти. Копия завершила свое выполнение досрочно, и этот факт привел к освобождению ресурсов памяти. Этот момент можно наблюдать на графике, где примерно половина общего объема памяти становится доступной после завершения копии. Далее процесс развивается согласно логике первого этапа.

\section{Эксперимент №2}

При $K=10$ процессы не прервутся так, как запущеные $newmem.sh$ успевают завершится, высвобождая память для новых запусков - ромежуток в секунду на самом деле достаточен для этого. Однако и при $K=30$ на моем устройстве процессы выполняются успешно. (даже физическая память не успевает заполнится). Думаю это связанно с железом на котором происходит тестирование. Поэтому эксперемент буду проводить при $K=50$ - OOM убил 10 процессов.

С помощью метода бинарного поиска можно найти $N$ при котором процессы завершатся успешно: \textbf{11662500}

\end{document}